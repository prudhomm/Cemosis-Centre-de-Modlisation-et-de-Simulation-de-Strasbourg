L'organisation de Cemosis

Sous la direction de Christophe Prud'homme, Cemosis se repose sur des chercheurs de l'équipe Modélisation et Contrôle de de l'IRMA de l'université de Strasbourg ainsi que sur des chercheurs de l'université de Haute-Alsace, dans le but de remplir ses missions. On compte environ 9 chercheurs permanents ayant participé aux activités de recherche pluridisciplinaires ou aux collaborations mathématiques-entreprises.

Depuis 2016, Cemosis possède aussi un ingénieur de Recherche dédié à la plateforme. Un project manager ainsi qu'un business développer sont venus ensuite	 soutenir l'activité des chercheurs de l'équipe.

Une des actions de Cemosis pour partager son travail à l'intérieur de ses collaborations est le développement de logiciel, principalement pour des méthodes numériques, des algorithmes, ainsi que pour quelques applications.
Le logiciel phare de Cemosis, Feel++ \url{http://www.feelpp.org} est une plateforme logiciel ouverte autour de laquelle se rassemblent chercheurs, utilisateurs et développeurs pour collaborer.
D'autres logiciels sont développés à l'intérieur de Cemosis, de part certaines collaborations. On notera AngioTK, pour la reconstruction de vaisseaux sanguins à partir d'images médicales, l'infrastructure cloud pour HPC MSO4SC ainsi que Selalib, une bibliothèque logicielle de méthodes numériques semi-lagrangienne pour la physique des plasmas.

Les collaborations avec Cemosis sont multiples, qu'elles soient pluridisciplinaires et académiques, ou avec une entreprise elles peuvent intervenir à plusieurs niveaux.

\subsection*{Premier niveau : les collaborations courtes}
Les outils de simulation dont nous avons la maîtrise sont directement applicables, des formations sont disponibles pour apprendre à utiliser ces outils. Une telle collaboration est courte (entre plusieurs mois et un an). En général une collaboration de ce type constitue pour nous l'occasion d'entamer des collaborations fructueuses sur le plus long terme.

\subsection*{Second niveau : collaborations moyennes}
Les modèles nécessaires à l'implémentation du partenariat existent, mais les outils logiciels ne sont pas disponibles et doivent être développés ou adaptés. Dans ce cas, il nous faut recruter un postdoctorat ou un ingénieur, et la collaboration dure entre un à deux ans.

\subsection*{Troisième niveau : Thèse}
Lorsque les modèles, méthodes et logiciels ne sont pas directement à notre disposition, il s'agit de projet de recherche ambitieux nécessitant au minimum une thèse. Que la thèse soit Cifre ou en contrat doctoral dans le cadre d'un projet pluridisciplinaire, la collaboration s'engage pour un minimum de 3 ans.
