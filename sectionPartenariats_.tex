\section*{Partenariats et success stories}

\subsection*{MSO4SC : Mathematical Modelling, Simulation \& Optimization for Societal Challenges with Scientific Computing}

La complexité croissante ainsi que le processus de résolution des problèmes de société nécessite que l'on procure des outils d'aide à la décision qui permettent de l'analyse de risque à long terme, de l'optimisation et même du contrôle ainsi qu'une amélioration au cours du temps.
Une des technologie clé de ce procédé est l'utilisation des méthodes MSO (Modélisation, Simulation et Optimisation). Ces méthodes ont prouvé leur efficacité pour résoudre des problèmes divers.

Cependant ces méthodes sont hautement complexes et nécessitent l'utilisation d'outils modernes de type calcul haute performance ou accès à des base de données big data, ce qui nécessite souvent l'aide d'experts qualifiés. Ces experts qualifiés sont souvent hors d'atteinte, par exemple dans les petites et moyennes entreprises.

Pour palier à ce problème, le réseau pan-européen EU-MATHS-IN (European Service Network of Mathematics for Industry and Innovation) a été fondé à partir de réseaux nationaux contenant les centres de recherche européen leader dans le domaine du MSO et de l'ICT (technologies de l'information et de la communication)

L'objectif principal du projet MSO4SC est de construire une e-infrastructure qui fournit un accès intégré, sur-mesure, et centré sur l'utilisateur vers les services, ressources et mêmes outils nécessaire au prototypage rapide. L'objectif est de fournir les services ainsi que les environnements de travail mathématiques.

La e-infrastructure consiste en un catalogue d'applications MSO, qui contiennent les modèles, logiciels, validation et benchmark ainsi que le MSOCloud : une infrastructure cloud facile d'utilisation pour les applications MSO sélectionnées et les environnements de travail développés dans le catalogue.
Ceci permet de résoudre le temps de mise sur le marcher des consultants travaillant dans les challenges de société sus-mentionnés.

\subsection*{Classification de courbes de charge}

Afin de mieux comprendre la consommation de chacun de ses clients, Électricité de Strasbourg a mis en place des appareils de mesure de courbes de chage. La charge de chaque client est calculée pendant 30 minutes et la courbe a été enregistrée. En croisant cette information avec la température extérieure, la typologie de client (individuel ou professionnel), le type de chauffage, il est possible de prévoir la charge du réseau.
Le but de la collaboration est l'obtention d'une classification non supervisée de la population.
Les formes et périodicité des courbes, corrélation entre différents jours ou dépendance entre la charge et les facteurs extérieurs sont autant de facteurs pris en compte pour réaliser la classification.

\grid