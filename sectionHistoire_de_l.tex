\section*{Histoire de la structure}
	
Cemosis a été créé en janvier 2013 à la suite de l'appel à projet IDEX attractivité 2012 pour accompagner l'arrivée de C. Prud'homme comme Professeur à l'IRMA dans l'équipe Modélisation \& Contrôle de P. Helluy. Il s'agissait d'une part d'ouvrir de nouveaux axes de recherche outre la fusion et la physique des plasmas, qui est au coeur de l'équipe, et de développer les relations entre les mathématiques et les entreprises d'autre part dans le cadre du Labex IRMIA. L'opportunité offerte par les financements IDEX et LABEX IRMIA et les besoins en recherche multi-disciplinaire et en particulier en modélisation, simulation et optimisation  à Strasbourg ont naturellement conduit à la création de Cemosis.

En France en 2013, il existait des structures locales et nationales pour la modélisation et la simulation. Au niveau local, nous pouvons citer MaiMoSiNE à Grenoble, dont C. Prud'homme a été un des fondateurs, CaSciMoDOT à Orléans ou encore ICS à Paris et au niveau national l'Agence des Mathématiques en Interaction avec l'Entreprise et la Société (AMIES) ou encore la Maison de la Simulation  (MdS). De telles structures (interdisciplinaires et/ou tournées vers l'industrie) ont été également créées à l'étranger: on pourra par exemple citer chez nos voisins allemands l'ICSC à Heidelberg, Matheon à Berlin ou encore AICES à Aachen.
Les mathématiciens sont à l'origine des toutes ces structures en France et à l'étranger et ils sont les forces vives une fois mises en place. 
En 2015, l'Étude sur l'Impact Socio-Économique des Mathématiques (EISEM) en France a dressé un premier bilan sur 
\begin{quote}
la  forte contribution des mathématiques à l'économie nationale et cette étude révèle/considère qu'elles joueront un rôle majeur pour relever les défis industriels et sociétaux de demain.
\end{quote}

Cemosis, au même titre que MaiMoSiNE, est cité dans ce rapport et est considéré comme une orientation à suivre pour accompagner les relations recherche industrie (pages 15 et 42). 
Ce rapport a joué un rôle incontestable pour faciliter les discussions et la promotion des mathématiques sur le site strasbourgeois tant auprès des collègues, que de l'institution, des étudiants mais aussi des entreprises.
En Novembre 2016, Cemosis recrute un Ingénieur de Recherche à durée indéterminée financé par l'Université de Strasbourg. Ce recrutement préfigure un tournant important pour notre structure.
Courant 2017, le réseau MSO a été créé sous l'impulsion d'AMIES: des structures ont émergées dans chaque région attestant des enjeux partagés par la communauté mathématique sur les relations avec les entreprises, la société ou encore les autres disciplines. 
Au delà même des enjeux, nous pourrions même parler des envies partagées.
Fin 2017, Cemosis accueille l'équipe en charge du projet Simseo à Strasbourg, Alsacalcul Services, dont le but est la valorisation de la simulation pour les petites et moyennes entreprises, afin de fournir une vitrine homogène de services autour de la simulation en Alsace.
Nous rédigeons à ce moment un plan stratégique de développement à 4 ans avec une transition pour Cemosis de devenir une plateforme de l'Université. 
Si Cemosis n'a pas le chiffre d'affaires de plateforme telles que le Sertit, le Service Régional de Traitement d'Image et de Télédétection, il est considéré comme ayant un fort potentiel par l'Université et est affichée en tant que telle sur dès l'ouverture du \href{http://entreprises.unistra.fr/innover/decouvrir-les-plateformes-technologiques-de-luniversite/#c3136}{portail entreprise de l'Université} en janvier 2018.

Si en 2010 avec la création de MaiMoSiNE, 2011 celle d'AMIES et en 2013 celle de Cemosis, le contexte de développement des relations mathématiques entreprises relevait davantage du constat de l'inexistence ou du manque de visibilité de l'interface entre mathématique et les entreprises, en 2017 il est tout autre! La révolution digitale des entreprises, l'avènement du HPC et du cloud computing, l'Internet des Objets, la fabrication additive sont des sujets brulants . Les mathématiques sont un facteur clé dans la réussite de ces développements dans le futur. Cemosis a décidé d'avoir une approche de la recherche orientée vers les applications,l'interdisciplinaire et les collaborations.
 
 

\vspace{1cm}